$\bm{C}$ --- матрица парных сравнений критериев, $\bm{\bm{A}}_i$ --- матрицы парных сравнений альтернатив в соответствии с критерием $i$:
$$\bm{C}= \begin{pmatrix}
1 & 1/5 & 1/5 & 1 & 1/3\\
5 & 1 & 1/5 & 1/5 & 1\\
5 & 5 & 1 & 1/5 & 1\\
1 & 5 & 5 & 1 & 5\\
3 & 1 & 1 & 1/5 & 1
\end{pmatrix},
$$
$$\bm{A}_1= \begin{pmatrix}
1 & 3 & 7 & 9\\
1/3 & 1 & 6 & 7\\
1/7 & 1/6 & 1 & 3\\
1/9 & 1/7 & 1/3 & 1
\end{pmatrix},
\qquad
\bm{A}_2= \begin{pmatrix}
1 & 1/5 & 1/6 & 1/4\\
5 & 1 & 2 & 4\\
6 & 1/2 & 1 & 6\\
4 & 1/4 & 1/6 & 1
\end{pmatrix},
$$
$$\bm{A}_3= \begin{pmatrix}
1 & 7 & 7 & 1/2\\
1/7 & 1 & 1 & 1/7\\
1/7 & 1 & 1 & 1/7\\
2 & 7 & 7 & 1
\end{pmatrix},
\qquad
\bm{A}_4= \begin{pmatrix}
1 & 4 & 1/4 & 1/3\\
1/4 & 1 & 1/2 & 3\\
4 & 2 & 1 & 3\\
3 & 1/3 & 1/3 & 1
\end{pmatrix},
$$
$$\bm{A}_5= \begin{pmatrix}
1 & 1 & 7 & 4\\
1 & 1 & 6 & 3\\
1/7 & 1/6 & 1 & 1/4\\
1/4 & 1/3 & 4 & 1
\end{pmatrix}.
$$
Нужные степени матрицы $\bm{C}$:
$$\bm{C}^2 = \begin{pmatrix}
1 & 5 & 5 & 1 & 5\\
5 & 1 & 1 & 5 & 5/3\\
25 & 5 & 1 & 5 & 5\\
25 & 25 & 5 & 1 & 5\\
5 & 5 & 1 & 3 & 1
\end{pmatrix},
\qquad
\bm{C}^3 = \begin{pmatrix}
25 & 25 & 5 & 1 & 5\\
5 & 25 & 25 & 5 & 25\\
25 & 25 & 25 & 25 & 25\\
125 & 25 & 5 & 25 & 25\\
25 & 15 & 15 & 5 & 15
\end{pmatrix},
$$
$$\bm{C}^4 = \begin{pmatrix}
125 & 25 & 5 & 25 & 25\\
125 & 125 & 25 & 5 & 25\\
125 & 125 & 125 & 25 & 125\\
125 & 125 & 125 & 125 & 125\\
75 & 75 & 25 & 25 & 25
\end{pmatrix},
\qquad
\bm{C}^5 = \begin{pmatrix}
125 & 125 & 125 & 125 & 125\\
625 & 125 & 25 & 125 & 125\\
625 & 625 & 125 & 125 & 125\\
625 & 625 & 625 & 125 & 625\\
375 & 125 & 125 & 75 & 125
\end{pmatrix}.
$$
Спектральный радиус матрицы $\bm{C}$:
$$\lambda_{\bm{C}} = \mathrm{tr}\bm{C}\oplus \dots \oplus \mathrm{tr}^{1/5}(\bm{C}^{5}) = (125)^{1/4} \approx 3.3437.$$
Матрица $\lambda^{-1}\bm{C}$ и ее степени:
$$(\lambda^{-1}\bm{C})^1 = \begin{pmatrix}
(1/125)^{1/4} & (1/78125)^{1/4} & (1/78125)^{1/4} & (1/125)^{1/4} & (1/10125)^{1/4}\\
(5)^{1/4} & (1/125)^{1/4} & (1/78125)^{1/4} & (1/78125)^{1/4} & (1/125)^{1/4}\\
(5)^{1/4} & (5)^{1/4} & (1/125)^{1/4} & (1/78125)^{1/4} & (1/125)^{1/4}\\
(1/125)^{1/4} & (5)^{1/4} & (5)^{1/4} & (1/125)^{1/4} & (5)^{1/4}\\
(81/125)^{1/4} & (1/125)^{1/4} & (1/125)^{1/4} & (1/78125)^{1/4} & (1/125)^{1/4}
\end{pmatrix},
$$
$$(\lambda^{-1}\bm{C})^2 = \begin{pmatrix}
(1/15625)^{1/4} & (1/25)^{1/4} & (1/25)^{1/4} & (1/15625)^{1/4} & (1/25)^{1/4}\\
(1/25)^{1/4} & (1/15625)^{1/4} & (1/15625)^{1/4} & (1/25)^{1/4} & (1/2025)^{1/4}\\
(25)^{1/4} & (1/25)^{1/4} & (1/15625)^{1/4} & (1/25)^{1/4} & (1/25)^{1/4}\\
(25)^{1/4} & (25)^{1/4} & (1/25)^{1/4} & (1/15625)^{1/4} & (1/25)^{1/4}\\
(1/25)^{1/4} & (1/25)^{1/4} & (1/15625)^{1/4} & (81/15625)^{1/4} & (1/15625)^{1/4}
\end{pmatrix},
$$
$$(\lambda^{-1}\bm{C})^3 = \begin{pmatrix}
(1/5)^{1/4} & (1/5)^{1/4} & (1/3125)^{1/4} & (1/1953125)^{1/4} & (1/3125)^{1/4}\\
(1/3125)^{1/4} & (1/5)^{1/4} & (1/5)^{1/4} & (1/3125)^{1/4} & (1/5)^{1/4}\\
(1/5)^{1/4} & (1/5)^{1/4} & (1/5)^{1/4} & (1/5)^{1/4} & (1/5)^{1/4}\\
(125)^{1/4} & (1/5)^{1/4} & (1/3125)^{1/4} & (1/5)^{1/4} & (1/5)^{1/4}\\
(1/5)^{1/4} & (81/3125)^{1/4} & (81/3125)^{1/4} & (1/3125)^{1/4} & (81/3125)^{1/4}
\end{pmatrix},
$$
$$(\lambda^{-1}\bm{C})^4 = \begin{pmatrix}
(1)^{1/4} & (1/625)^{1/4} & (1/390625)^{1/4} & (1/625)^{1/4} & (1/625)^{1/4}\\
(1)^{1/4} & (1)^{1/4} & (1/625)^{1/4} & (1/390625)^{1/4} & (1/625)^{1/4}\\
(1)^{1/4} & (1)^{1/4} & (1)^{1/4} & (1/625)^{1/4} & (1)^{1/4}\\
(1)^{1/4} & (1)^{1/4} & (1)^{1/4} & (1)^{1/4} & (1)^{1/4}\\
(81/625)^{1/4} & (81/625)^{1/4} & (1/625)^{1/4} & (1/625)^{1/4} & (1/625)^{1/4}
\end{pmatrix}.
$$
Матрица Клини:
$$(\lambda^{-1}\bm{C})^* = \bm{I} \oplus (\lambda^{-1}\bm{C})^1 \oplus (\lambda^{-1}\bm{C})^2 \oplus (\lambda^{-1}\bm{C})^3 \oplus (\lambda^{-1}\bm{C})^4 = $$
$$ = \begin{pmatrix}
1 & (1/5)^{1/4} & (1/25)^{1/4} & (1/125)^{1/4} & (1/25)^{1/4}\\
(5)^{1/4} & 1 & (1/5)^{1/4} & (1/25)^{1/4} & (1/5)^{1/4}\\
(25)^{1/4} & (5)^{1/4} & 1 & (1/5)^{1/4} & (1)^{1/4}\\
(125)^{1/4} & (25)^{1/4} & (5)^{1/4} & 1 & (5)^{1/4}\\
(81/125)^{1/4} & (81/625)^{1/4} & (81/3125)^{1/4} & (81/15625)^{1/4} & 1
\end{pmatrix}.
$$
Линейно независимые столбцы:
$$\bm{P} = \begin{pmatrix}
1 & (1/25)^{1/4}\\
(5)^{1/4} & (1/5)^{1/4}\\
(25)^{1/4} & (1)^{1/4}\\
(125)^{1/4} & (5)^{1/4}\\
(81/125)^{1/4} & 1
\end{pmatrix}.
$$
Лучший и худший дифференцирующие векторы весов критериев.
$$\bm{w}_1 = \begin{pmatrix}
(1/125)^{1/4}\\
(1/25)^{1/4}\\
(1/5)^{1/4}\\
(1)^{1/4}\\
(81/15625)^{1/4}
\end{pmatrix},
\qquad \bm{w}_2 = \begin{pmatrix}
(1/125)^{1/4} & (1/125)^{1/4}\\
(1/25)^{1/4} & (1/25)^{1/4}\\
(1/5)^{1/4} & (1/5)^{1/4}\\
(1)^{1/4} & (1)^{1/4}\\
(1/125)^{1/4} & (1/5)^{1/4}
\end{pmatrix}.
$$
Взвешенные суммы матриц парных сравнений векторов совпали:
$$\bm{B} = \bm{D} =\begin{pmatrix}
(1)^{1/4} & (2401/5)^{1/4} & (2401/5)^{1/4} & (6561/125)^{1/4}\\
(25)^{1/4} & (1)^{1/4} & (1296/125)^{1/4} & (81)^{1/4}\\
(256)^{1/4} & (16)^{1/4} & (1)^{1/4} & (81)^{1/4}\\
(81)^{1/4} & (2401/5)^{1/4} & (2401/5)^{1/4} & (1)^{1/4}
\end{pmatrix}.
$$

Спектральный радиус матрицы $\bm{B}$:
$$\lambda_{\bm{B}} = \mathrm{tr}\bm{B}\oplus \dots \oplus \mathrm{tr}^{1/4}(\bm{B}^{4}) = (614656/5)^{1/8} \approx 4.32721.$$
Матрица $\lambda^{-1}\bm{B}$:
$$\lambda^{-1}\bm{B} = \begin{pmatrix}
(5/614656)^{1/8} & (2401/1280)^{1/8} & (2401/1280)^{1/8} & (43046721/1920800000)^{1/8}\\
(3125/614656)^{1/8} & (5/614656)^{1/8} & (6561/7503125)^{1/8} & (32805/614656)^{1/8}\\
(1280/2401)^{1/8} & (5/2401)^{1/8} & (5/614656)^{1/8} & (32805/614656)^{1/8}\\
(32805/614656)^{1/8} & (2401/1280)^{1/8} & (2401/1280)^{1/8} & (5/614656)^{1/8}
\end{pmatrix}.
$$
Матрица Клини:
$$(\lambda^{-1}\bm{B})^* = \bm{I} \oplus (\lambda^{-1}\bm{B})^1 \oplus (\lambda^{-1}\bm{B})^2 \oplus (\lambda^{-1}\bm{B})^3 = $$
$$ = \begin{pmatrix}
1 & (2401/1280)^{1/8} & (2401/1280)^{1/8} & (6561/65536)^{1/8}\\
(32805/614656)^{1/8} & 1 & (6561/65536)^{1/8} & (32805/614656)^{1/8}\\
(1280/2401)^{1/8} & (1)^{1/8} & 1 & (32805/614656)^{1/8}\\
(1)^{1/8} & (2401/1280)^{1/8} & (2401/1280)^{1/8} & 1
\end{pmatrix}.
$$
Линейно независимые столбцы:
$$\bm{P} = \begin{pmatrix}
1 & (2401/1280)^{1/8} & (6561/65536)^{1/8}\\
(32805/614656)^{1/8} & 1 & (32805/614656)^{1/8}\\
(1280/2401)^{1/8} & (1)^{1/8} & (32805/614656)^{1/8}\\
(1)^{1/8} & (2401/1280)^{1/8} & 1
\end{pmatrix}.
$$
Лучший дифференцирующий вектор:
$$\bm{w}_{best} = \begin{pmatrix}
1 & (6561/65536)^{1/8}\\
(32805/614656)^{1/8} & (32805/614656)^{1/8}\\
(1280/2401)^{1/8} & (32805/614656)^{1/8}\\
(1)^{1/8} & (1)^{1/8}
\end{pmatrix}\bm{v}_{1},
\qquad
\bm{v}_{1}
>
\bm{0}.
$$
Худший дифференцирующий вектор.
$$
\qquad \bm{w}_{worst} = \begin{pmatrix}
1\\
(1280/2401)^{1/8}\\
(1280/2401)^{1/8}\\
(1)^{1/8}
\end{pmatrix}\bm{v}_{2},
\qquad
\bm{v}_{2}
>
\bm{0}.
$$
Ответ:
$$\bm{w}_{best} \approx \begin{pmatrix}
1.000000 & 0.750000\\
0.693288 & 0.693288\\
0.924384 & 0.693288\\
1.000000 & 1.000000
\end{pmatrix}\bm{v}_{1},
\qquad
\bm{v}_{1}
>
\bm{0},
\qquad
\bm{w}_{worst} \approx \begin{pmatrix}
1.000000\\
0.924384\\
0.924384\\
1.000000
\end{pmatrix}\bm{v}_{2},
\qquad
\bm{v}_{2}
>
\bm{0}.
$$
